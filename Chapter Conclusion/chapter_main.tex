\chapter{Conclusion}
\label{chap:conclusion}
%\addcontentsline{toc}{chapter}{Conclusion}

For many decades, secure computation researchers have focused on developing privacy-enhancing technologies, working their way from inefficient, but fascinating, solutions to practical realizations adopted in the real world. Due to increased interest in the field and improvements in computational power, we now find ourselves in an era where the question is no longer if it can be done but whether if it is feasible. 

The cooperation between the security services is essential to national security, but sharing classified intelligence is not straightforward due to security concerns and its potential infringement of the rights of the natural person. We considered the case of the \acrshort{pnr} registry, where an apparent deadlock occurs when a client wants to perform a query, containing classified information, on the registry but is incapable of doing so as it cannot share the classified information. Such a case inhibits the purpose of the registry and limits cooperation.

We showed that this apparent deadlock is resolvable with the use of \acrshort{mpc}, demonstrating that one of our proposed solutions can support a megabyte-size database. Further, we approximate that another proposed solution should improve those results to the order of gigabytes. These solutions not only preserve the privacy of the involved parties, allowing them to take full advantage of the registry, but also increase the capability of searches on the registry, as demonstrated with the keyword search protocol. These feasible results are enabled by \acrshort{pds}, which facilitates combining \acrshort{psi} protocols with \acrshort{ot} protocols to empower various types of searches and protocol properties. With our implementation we show that both a keyword search and semantic search are possible, but we remark that only the keyword search is feasible.

A question left open is the implication of a \acrshort{wlan} setting on the results; practitioners must consider that even a low network latency might not be sufficient. For example, our proposed \acrshort{ot} protocol, \cref{sec:databaseinitiation} and \cref{sec:recordretrieval}, for record retrieval requires a quasilinear number of messages exchanged between the client and server, thus making even a 1 ms network latency too long for large database sizes. Therefore, future work should consider using that state-of-the-art \acrshort{spir} protocol XSPIR \cite{ESORICS:LinLiuMal22} for the retrieval of records, to better understand the database sizes that can be supported.

Further, we emphasize the insightful observation that $ \Encode $ and $ \Retrieve $ needs to exert non-deterministic semantic security in order to allow for multi-tenancy and that the communication flow between the client and server was not captured in our privacy definition to suggest replacing the game-based approach with a simulation-based approach as it would better allow for describing the ideal functionality of \acrshort{pds} protocols, something that is key in steps towards malicious security.

All things considered, it is exciting to see that \acrshort{mpc} is feasible for private searches on private databases, as similar problems are prevalent in processes that consider a database with personal data or classified information. We remark however that in the case of the \acrshort{pnr} registry the additional involvement of jurists and others is necessary for defining the right requirements surrounding suitable solutions.

From the proof of concept, we found that for a megabyte-sized database, it takes hours to complete a single transfer, therefore, it would be beneficial to structure our protocol so that we do most of the costly work in advance. We encode our database $ D $ in the initiation to make the transfers as efficient as possible. The overarching idea is for the client to obliviously encrypt and shuffle $ D $ with its keys $ K $ and permutation $ \pi $. If we were to follow a general \acrshort{oram} construction, the client would non-obliviously encrypt the records and use an oblivious sorting algorithm to shuffle them. This process would only provide privacy in one way; hence, we introduce the notion of \acrfull{ps}, which yields privacy for both the client and server in the shuffling and encryption of the database. For \acrshort{ps}, we present an efficient generic \acrshort{mpc} protocol that uses binary circuits. We use a generic \acrshort{mpc}-based \acrshort{ps} protocol in combination with an oblivious sorting algorithm to implement the $ \PreProcess $ algorithm. With the implementation, we perform experiments to find the largest database size we can support within a time budget $ \tau $ of eight hours. After presenting the results from the experiment, we propose an \acrshort{ot}-based \acrshort{ps} protocol and discuss its potential efficiency.
We have initiated the \acrshort{ot} protocol, so let us move to the transfer phase. Recall that we are interested in encoding the indices that the client gets from the search so that they can be used to retrieve the relevant records without leaking information about which records were transferred. This is done using what we did in $ \PreProcess $ to facilitate the functionality of the $ \Encode $, $ \Retrieve $ and $ \Decode $ algorithms. As done previously, we implement and perform experiments with these algorithms and then discuss their efficiency.

\refstepcounter{subsection}\subsection*{\thesubsection\quad The $ \Encode, \Retrieve $ and $ \Decode $ algorithms. }\label{subsec:FileRetrieval}

The client wants to encode the set of indices $ I $ that it acquired in the search. We realize that $ \pi $ is a permutation of the original indices of $ D $ and, therefore, works as a translation between the indices used to derive the auxiliary information $ A $ and the indices of the encoded database $ D' $. This means that we can use the permutation $ \pi $ to encode the indices $ I $ so that they are unlinkable to any specific record, in the perspective of the server. Imagine that we are interested in index $ \iota $; to encode it, we take the $ \iota^{\text{th}}$ index of the permutation, $ \iota' \gets \pi_\iota $. This way, we can build the encoded indices $ I' $, \cref{algo:Encode}.

\hfill

\RestyleAlgo{ruled}
\begin{algorithm}[H]
    \LinesNumbered
    \caption{Client - $ \Encode $}
    \label{algo:Encode}
    \KwIn{Record indices $ I $}
    \KwIn{Dummy record indices $ \eta $}
    \KwIn{Permutation $ \pi $}
    \KwIn{Requested indices $ \upsilon $}
    \KwOut{Encoded indices $ I' $}

    \vspace*{0.48cm}

    $ I' = \left[ \phantom{=} \right] $

    \For{$ \iota \mathrm{\phantom{|} in \phantom{|}} I $}{

        \If{$ \iota \mathrm{\phantom{|} in \phantom{|}} \upsilon $}{

            $ \iota \overset{{\scriptscriptstyle\$}}{\gets} \eta $

            $ \eta $.remove$ \left( i \right) $
        }

        $ i' \gets \pi_\iota $

        $ I' $.add$ \left( i' \right) $
    }

    \For{$ \iota \in \lbrace \left| I' \right|, ... , \sigma \rbrace $}{

        $ \iota \overset{{\scriptscriptstyle\$}}{\gets} \eta $

        $ \eta $.remove$ \left( \iota \right) $

        $ i' \gets \pi_\iota $

        $ I' $.add$ \left( i' \right) $

    }

    $ \upsilon $.addAll$ \left( I' \right) $

    \vspace*{0.48cm}

    \KwRet{$ I' $}
\end{algorithm}

\clearpage

Further, we realize that different lengths of $ I $ could be used as a direct identifier for the results from the search. Therefore, we always want the client to request the same amount of records, and to achieve this, we pad $ I $ with a fixed length $ \sigma $ while never requesting the same index twice. 

Recall that $ \eta $ contains the indices of the dummy records, meaning that we can use them to pad $ I $. It is also important to realize that if an index from $ I $ already has been requested, then it is unnecessary to request it again, so the client simply requests a dummy record in its place. Once an index of a dummy record is requested, it is removed from $ \eta $ and can, therefore, no longer be used. We keep track of these requested indices with the set $ \upsilon $.
\raggedbottom

The indices are now encoded, and the client sends them to the server. Once received, the server picks the encrypted record from $ D' $ for all the encoded indices, $ F' \gets \Retrieve \left( I' \right) $, \cref{algo:Retrieve}.

\hfill

\RestyleAlgo{ruled}
\begin{algorithm}[H]
    \LinesNumbered
    \caption{Server - $ \Retrieve $}
    \label{algo:Retrieve}
    \KwIn{Encoded Database $ D' $}
    \KwIn{Encoded indices $ I' $}
    \KwOut{Encrypted Records $ F' $}

    \vspace*{0.48cm}

    $ F' = \left[ \phantom{=} \right] $

    \For{$ \iota' \mathrm{\phantom{|} in \phantom{|}} I' $}{

        $ F' $.add$ \left( D'_{\iota'} \right) $
    }

    \vspace*{0.48cm}

    \KwRet{$ F' $}
\end{algorithm}

\hfill

Next, the server sends the encrypted records $ F' $ to the client, which uses the encryption keys $ K $ to decrypt the records by fetching key $ K_{\iota'} $ to decrypt record $ R_{\iota'} $ from $ F' $. We discard any received dummy records $ d_{\iota'} $. Giving us the implementation of $ \Decode $, \cref{algo:Decode}.

\hfill

\RestyleAlgo{ruled}
\begin{algorithm}[H]
    \LinesNumbered
    \caption{Client - $ \Decode $}
    \label{algo:Decode}
    \KwIn{Encrypted records $ F' $}
    \KwIn{Encryption keys $ K $}
    \KwOut{Transferred records $ F $}

    \vspace*{0.48cm}

    $ F = \left[ \phantom{=} \right] $

    \For{$ R_{\iota'} \mathrm{\phantom{|} in \phantom{|}} F' $}{

        $ R_\iota \gets \Decrypt \left( R_{\iota'}, K_{\iota'} \right) $

        $ F $.add$ \left( R_{\iota} \right) $
    }

    \vspace*{0.48cm}

    \KwRet{$ F $}
\end{algorithm}

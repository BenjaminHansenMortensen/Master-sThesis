The second type of search we present is a semantic search that uses a \acrshort{llm} to capture the semantics of objects by describing them as vector embeddings. To compare objects, we use the distance between them to describe their semantic likeness, meaning that a shorter distance implies similarity. In our case, we want the comparison to be done without the server learning anything about the search query $ q $ and the client only learning the set of indices $ I $ of the records with a distance shorter than some threshold $ t $. Compared to the keyword search, this type of search is not a straightforward \acrshort{psi} protocol as we are not asserting whether two objects are equal but if they are similar. Therefore, we use generic \acrshort{mpc} to compute the distance between elements. We split the protocol into an initiation and transfer phase. In the initiation, the server creates vector embeddings of all the records and store them in an indexing $ A $. In the transfer phase, the client and server compute the distance between every vector in $ A $ and the embedding of the search query in \acrshort{mpc}. The output is the set of indices $ I $, which contains the index of every record with a distance shorter than some threshold $ t $ to $ q' $. In the end, we implement the protocol and discuss the experimental results.


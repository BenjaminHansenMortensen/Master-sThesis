\newcommand{\PreProcess}{\ensuremath{\mathsf{{PreProcess}}}}
\newcommand{\Encrypt}{\ensuremath{\mathsf{{Encrypt}}}}
\newcommand{\Permute}{\ensuremath{\mathsf{{Permute}}}}
\newcommand{\Sort}{\ensuremath{\mathsf{{Sort}}}}
\newcommand{\Decrypt}{\ensuremath{\mathsf{{Decrypt}}}}
\newcommand{\Distance}{\ensuremath{\mathsf{{Distance}}}}
\newcommand{\KeyGen}{\ensuremath{\mathsf{{KeyGen}}}}
The general idea is to present solutions to the five algorithms $ \Filter $, $ \Search $, $ \Encode $, $ \Retrieve $ and $ \Decode $ such that they satisfy our description of privacy, correctness, adaptive, storage imbalance, access control and timely. \cref{fig:CommunicationFlow} outlines, but does not limit, the communication flow between the client and server in the five algorithms, therefore we introduce additional algorithms to realize the protocol.

We do not focus on the $ \Setup $ algorithm, but we highlight that $ \Setup $ shuffles the order of the records on the database. Recall that \acrshort{pds} consists of a \acrshort{psi} protocol, to facilitate the search, combined with an \acrshort{ot} protocol, to facilitate the retrieval. We propose two \acrshort{psi} protocols that provide the functionality of a keyword search and semantic search on the database, and one \acrshort{ot} protocol that allows retrieval of records from the database. However, the two types of searches are independently realized so a \acrshort{pds} protocol might only include one, therefore, we present both as their own stand-alone protocol. Overall, the protocol is split into an initiation step of the subprotocols and a transfer step where the protocol is executed. To realize the protocol we extend the work in \cref{chap:PDS} and introduce eight new algorithms and re-introduce $ \Garble $:

\begin{description}
    \item[$ \PreProcess $] takes in the database $ D $, and outputs the set of dummy item indices $ \eta $, a permutation $ \pi \subseteq {\lbrace 1, 2, ... , n \rbrace} $ of length $ n $, encryption keys $ K $, and the encoded database $ D' $ with the files encrypted and shuffled.
    \item[$ \Encrypt $] takes in a plaintext and encryption key of length $ \beta $, and outputs a ciphertext of length $ \beta $.
    \item[$ \Garble $] takes in some auxiliary information $ A $, and outputs the encrypted auxiliary information $ A' $.
    \item[$ \Permute $] takes in an array, and outputs a permutation $ \pi $ of that array.
    \item[$ \Sort $] takes in a permutation $ \pi $, and outputs a series of indices $ i, j $ and a bit $ b $ that indicates whether the indices should be swapped or not.
    \item[$ \Hash $] takes in some binary string and outputs a digest of length $ \beta $.
    \item[$ \Decrypt $] takes in a ciphertext and encryption key of length $ \beta $, and outputs a plaintext of length $ \beta $.
    \item[$ \Distance $] takes in two vectors and calculates the distance between them. 
    \item[$ \KeyGen $] takes in no input but outputs an encryption key and other variables like IVs, nonces and counters.
\end{description}

For the sake of brevity but without the lack of generality, we abstract some detailed functionalities into a general \acrshort{mpc} description to best communicate the overall idea behind the algorithms.
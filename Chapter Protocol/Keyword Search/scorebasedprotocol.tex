\refstepcounter{subsection}\subsection*{\thesubsection\quad Score-based Search}\label{subsec:KeywordSearchScoreBased}

The inverted index matrix $ A $ is not limited to two values per tuple; we can add additional elements to include new information related to the search. Imagine that we have a scoring system consisting of  grades; each refers to different variants of our database where the details in the records vary. The challenge is ensuring that the client retrieves records from the correct database. A neat property of \acrshort{pds} is that the search and retrieval are independent; therefore, if the client performs a search and learns what grade it is qualified for, it can then send that grade to the server to gain access to the corresponding variant of the database. In this way, we further empower the server to enforce better access control of the database.

To include the grades we add a third column in $ A $ with the grade corresponding to each keyword $ v_j $. The encryption follows the same idea as with the encryption of the indices. We pick a new encryption key $ e_3 $ that is combined with $ v_j $ to derive an ephemeral key $ e_{3,j} \gets \Encrypt\left(v_j, e_3\right) $, that encrypts the grade in row $ j $. In the transfer phase, the client and server require an additional \acrshort{oprf} execution for the client to get its search query $ q $ encrypted under $ e_3 $, so that it can decrypt the grade in the row it is interested in. Once it has the grade, it sends it to the server, which grants it access to the corresponding database variant, and then the client continues by retrieving the relevant records.

There are challenges with this score-based search as the grades can work as direct identifies, and supporting multiple variants of the database brings with it the question of feasibility.

\refstepcounter{subsection}\subsection*{\thesubsection\quad Post-Quantum Security}\label{subsec:DiscussionPostQuantumSecurity}

One of this work's main objectives is to avoid using post-quantum insecure hardness assumptions. However, the program used in MP-SPDZ to perform generic \acrshort{mpc} in the experiments was semi-party.x, which denotes a stripped-down version of the MASCOT \cite{CCS:KelOrsSch16} protocol. MASCOT performs computations with arithmetic circuits, that uses Beaver triples \cite{C:Beaver91b}. These triples are distributed among the parties with \acrshort{ot}, and even though most of the \acrshort{ot}s are realized with symmetric primitives through \acrshort{ot}-extension, some base \acrshort{ot}s that use asymmetric encryption are still required. Given the results from the work of Boyle et al. \cite{CCS:BCGIKRS19}, it is fair to assume that the base \acrshort{ot}s can be performed to the same magnitude of efficiency with a post-quantum secure hardness assumption. This suggests that the magnitude of efficiency presented in this work still holds. 

In the keyword search, we implemented an \acrshort{oprf} using generic \acrshort{mpc}, that can be replaced with a post-quantum secure ORPF \cite{PKC:ADDS21}. 

Regarding the \acrshort{ot}-based \acrshort{ps} protocol, the results presented were comparable to the efficiency of the post-quantum secure \acrshort{ot} protocol from Boyle et al. mentioned above. Thus, its results still seem promising. This is important as the \acrshort{ps} protocol used in the experiments used generic \acrshort{mpc} and still yielded much worse performance than we approximated the \acrshort{ot}-based \acrshort{ps} protocol to yield. \footnote{The semantic search utilizes generic \acrshort{mpc} but is excluded due to its poor experimental results.}
By Norwegian law, the different Norwegian security services are only allowed to investigate and collect types of intelligence within the scope
of their jurisdiction. However, intelligence gathered by the other security services could be valuable depending on the investigation. The problem arises when the intelligence collected in an investigation is classified, and the potentially critical intelligence gathered by another security service is also classified, leading to an apparent deadlock where intelligence is either not shared or unlawfully shared. Together, we examine the case of the \acrfull{pnr} registry and show that such an apparent deadlock is avoidable. We then give a thesis statement, share the contributions of this work, and introduce the experimental setup.

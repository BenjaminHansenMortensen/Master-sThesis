\section{Oblivious Transfer}
\label{sec:ot}
%\addcontentsline{toc}{section}{Oblivious Transfer}

\acrshort{ot} protocols come in many variants; the first variant of \acrshort{ot}, referred to as Rabin \acrshort{ot}, was introduced by Michael Oser Rabin \cite{EPRINT:Rabin05}. It functions differently from what we usually think of today as \acrshort{ot}, as it gives the receiver a 50/50 chance of receiving some message from a sender without the sender learning if the message was received. Today, the common intuition is that the receiver can pick some elements from a more extensive set in possession of the sender. The sender then transfers those elements to the receiver without learning which elements were transferred. In its most basic form, $\text{\acrshort{ot}}^2_1$ \cite{C:EveGolLem82}, a receiver can learn one of two messages without the sender learning which message the receiver chose. From $\text{\acrshort{ot}}^2_1$, one can build other variants of \acrshort{ot}, like $\text{\acrshort{ot}}^n_k$ \cite{IshaiK97}, which allows the receiver to choose $ k $-out-of-$ n $ elements. It was shown that \acrshort{ot} requires asymmetric assumptions for a black-box construction in the standard model \cite{FOCS:GKMRV00}, which is why \acrshort{ot}-extension \cite{STOC:Beaver96a} is essential for extending $\text{\acrshort{ot}}^2_1$ to $\text{\acrshort{ot}}^n_1$ and $\text{\acrshort{ot}}^n_k$. \acrshort{ot}-extension works by utilizing a few slow base \acrshort{ot}s from asymmetric assumptions to achieve many efficient symmetric \acrshort{ot}s. This is similar to the standard approach to secure communication, where slow asymmetric encryption is initially used to derive a standard shared key, followed by fast symmetric encryption for the remainder of the encryption of messages. 

An introduction to \acrshort{ot} would not be complete without mentioning that \acrshort{ot} is complete for secure computation \cite{STOC:Kilian88}, meaning that generic \acrshort{mpc} can be realized from $\text{\acrshort{ot}}^2_1$, making \acrshort{ot} one of the most fundamental research topics in secure computation. Considering the functionality of \acrshort{ot}, it most definitely would be used in our structure to retrieve records, as \acrshort{psi} would facilitate the search obliviously.

Regarding our problem, we want the functionality to be intuitive and natural to the end user. By that, we want the functionality to resemble modern search engines. In terms of efficiency, we want to perform successive efficient searches based on what we learned from the previous search. An \acrshort{ot} protocol with such a property is called an adaptive \acrshort{ot} protocol \cite{C:NaoPin99}. Further, depending on the identity of the user, the server could want to control the data that is accessible to that specific user. \acrshort{ot} protocols with this property are said to have access control \cite{CCS:CamDubNev09}. From the literature, we identify the protocol by Libert et al. \cite{AC:LLMNW17} as a potential candidate solution. However, they do not show experimental results, and there are questions raised about efficiency \cite{SCN:CDLNT20}.

Recall that \acrshort{pir} assumes a public database, and subsequently the client is assumed to know which records it is after. There is a variant called \acrshort{spir} \cite{STOC:GIKM98} that additionally provides privacy for the server, and therefore \acrshort{spir} is a special type of \acrshort{ot} protocol. The key element differentiating \acrshort{spir} as a class of protocols is the insistence on the non-trivial property. \acrshort{spir} is, therefore, a good candidate subfield to find solutions to our problem; however, as argued by Freedman et al., ``Given that KS allows clients to input an arbitrary search word, as opposed to selecting $p_i$ by an input $i$, keyword search is strictly stronger than the better-studied problems of oblivious transfer (OT) and symmetrically private information retrieval (SPIR).'' \cite{TCC:FIPR05} To the author's knowledge, a suitable post-quantum secure, adaptive keyword search protocol with access control is still not yet present in the literature.

\acrfull{pdq} \cite{ACNS:BGHWW13} is a generalization of \acrshort{spir}. It allows a client to submit complex queries to a database to retrieve records obliviously. An example of a complex query could be to ask for all records where the total number of people without a listed phone number is over a threshold. Mainly, \acrshort{pdq} aims to extend \acrshort{spir} to allow for SQL-like query functionality. Relating to our problem, this functionality is most likely desirable compared to regular search functionality. However, the problem arises when considering that such a solution requires the end user to know, for the sake of argument, SQL in order to perform searches. We cannot assume this to be the case, and given the use case, such a functionality would be overkill. Thus, to the best of the author's knowledge, there is no \acrshort{pdq} protocol present in the literature that, on its own, offers a sufficiently simplistic query interface for non-programmer users.
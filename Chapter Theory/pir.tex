\section{Private Information Retrieval}
\label{sec:pir}
%\addcontentsline{toc}{section}{Private Information Retrieval}

In the information-theoretic setting, \acrshort{pir} was introduced in the works of Eyal Kushilevitz, Oded Goldreich, and Madhu Sudan \cite{FOCS:CGKS95}. It lets clients retrieve records from a public database privately without the servers learning which records were retrieved. A trivial solution that achieves the same functionality is for the server to send the whole database to the client; therefore, \acrshort{pir} has non-trivial requirements, which say that the communication cost should be sublinear in the database size. This is a neat property as it explicitly respects that the client's storage is less than the server's, and implicitly, protocol designers focus on having the client do less computational work than the servers. In \acrshort{pir}'s introduction, replicating the database across multiple servers was used. The client would then secret share its query to retrieve individual shares of the data from the servers; thus, each server never learns the data retrieved by the client, assuming that the servers do not collaborate. In the computational setting, Rafail Ostrovsky and Eyal Kushilevitz showed that \acrshort{pir} could be achieved with a single server \cite{FOCS:KusOst97}. \acrshort{pir} protocols can, therefore, be realized with one or more servers, but implicitly, they should support multiple clients (see \cref{subsec:DiscussionPrivacy}).  

For our problem, recall that we are in the two-party setting; therefore, assuming the presence of an adversary, an honest party would never be in the majority. Implying that if the database were to be replicated across several servers, effectively introducing more parties, the assumption that these servers do not collaborate to break the privacy in the protocol is equivalent to assuming a third mutually trusted party. To see why let us imagine that the database, in possession of party $ P_2 $, is split into server $ \zeta_1 $ and $ \zeta_2 $. We now have three parties: the client $ P_1 $ and servers $ \zeta_1 $ and $ \zeta_2 $. From the client's perspective, we want to guarantee protection against semi-honest adversaries. To give a proof by contradiction, we assume an honest majority, implying that either $ \zeta_1 $ or $ \zeta_2 $ are honest, but as they are derived by $ P_2 $, the only way for this to be possible is for either of the servers to reside with a third independent party $ P_3 $. This party has to be trusted by $ P_2 $; if not, $ P_1 $ and $ P_3 $ could collaborate to break its privacy and contradict our assumption of an honest majority. Hence, $ P_3 $ has to be an independent mutually trusted third party, which would make the use of \acrshort{mpc} redundant as the core of the \acrshort{mpc} paradigm is to facilitate this functionality in the lack of a third independent mutually trusted party. This shows that, indeed, we cannot have an honest majority, and therefore, it prevents us from replicating the database across more than one server.

Presume we add \acrshort{pir} to our structure, we can then use \acrshort{psi} to have the client perform private searches on the database, and then use \acrshort{pir} to retrieve the records from the server. This only works if the underlying \acrshort{pir} protocol does not reveal any information about the database other than precisely the records the client is after.
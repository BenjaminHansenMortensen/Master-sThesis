Ideally, we hope to find in the literature a perfect candidate solution that optimally solves our problem, which is easier said than done, so we should be methodical about our search. The area of research where such a solution would be present is in the domain of \acrshort{mpc}, an area in cryptology that focuses on secure computations. \acrshort{mpc} is very broad and thus consists of many subfields focusing on specific computations. We explore the subfields relevant to our problem, narrowing the scope as we go, starting with exploring the subfields of \acrfull{oprf} and \acrfull{psi}, as combining the two gives us a structure to assess different parts of potential solutions. Given developments in other subfields, we suspect that swapping out the latter part of the structure with other methods of retrieving records could yield a more functional solution, therefore, we also examine the subfields of \acrfull{oram} and \acrshort{pir}. \acrshort{oram} would allow us to read and write data, which is neat as the database could continuously be updated with new records even after the initiation. On the other hand \acrshort{pir} intrinsically respects the imbalance between a client and a server, which is a useful aspect as only the server could require excessive hardware costs while weaker clients could easily be added. Inherently, the \acrshort{psi} and \acrshort{oram}/\acrshort{pir} structure falls under a different subfield \acrshort{ot}, so we naturally explore that as well. The goal is to understand the nuances between the subfields and identify attributes our solution should exert. Therefore, the discussion outlining what acceptable solutions to our problem is will be integrated with the theory discussion and is had when seemed fit.
